\chapter{Лабораторные работы}

\section{Массивы}

\subsection{Лабораторная работа 1. Одномерные массивы}
\begin{enumerate}[leftmargin=*]
    \item Дан массив вещественных чисел. Найдите сумму отрицательных элементов массива.
    \item Найдите произведение элементов массива с нечетными номерами.
    \item Дан массив целых чисел. Количество запросить с клавиатуры. Найти максимальный (минимальный) элемент массива и его номер, при условии, что все элементы различные.
    \item Найдите наименьший четный элемент массива. Если такого нет, то выведите первый элемент.
    \item Преобразовать массив так, чтобы сначала шли нулевые элементы, а затем все остальные.
    \item Ввести массив, в котором только два одинаковых элемента. Определить их местоположение.
    \item Ввести два массива действительных чисел. Определить максимальные элементы в каждом массиве и поменять их местами.
    \item Задан целочисленный массив. Определить процентное содержание элементов, превышающих среднеарифметическое всех элементов массива.
    \item Выполнить сортировку массива по возрастанию (убыванию).
    \item Дан массив из 10 элементов. Первые 4 упорядочить по возрастанию, последние 4 по убыванию.
\end{enumerate}
\subsection{Лабораторная работа 2. Сортировка и упорядочение массивов}
\begin{enumerate}[leftmargin=*]
    \item Создайте матрицу случайных чисел размерности $n \times m$ в диапазоне $[1;10]$.
    \item Дана квадратная матрица. Вывести на экран элементы, стоящие на диагонали.
    \item Дана матрица. Вывести на экран все нечетные столбцы, у которых первый элемент меньше последнего.
    \item Дана матрица $N\times M$ случайных чисел. Отсортировать элементы главной диагонали матрицы по убыванию.
    \item Дана матрица $N\times M$ случайных чисел. Упорядочить первый столбец матрицы по возрастанию, а последний столбец --- по убыванию.
    \item Дан массив из 10 элементов. Отсортируйте отдельно элементы от 0-го по 2-й, с 3-го по 5-й и с 6-го по 9.
    \item Дан трехмерный массив $N\times M\times K$ случайных чисел $(N, M, K>5)$. Отсортируйте матрицу $N\times M$ при $K=2$ и выведите её на экран монитора.
    \item Дан массив 20 целых чисел на отрезке $[-2;5]$. Упорядочить массив, удалив нули со сдвигом влево, ненулевыми элементами.
    \item Дан массив 20 целых чисел на отрезке $[-5;5]$. Упорядочить массив, удалив повторяющиеся элементы.
    \item Дан массив. Найдите два соседних элемента, сумма которых минимальна.
    \item В данном массиве найдите количество чисел, соседи у которых отличаются более чем в 2 раза.
\end{enumerate}

\section{Файлы}
\subsection{Лабораторная работа 1}
\begin{enumerate}[leftmargin=*]
    \item Создайте матрицу \monobf{x[n][n]} случайных чисел. Сохраните все элементы матрицы в файл с названием \monobf{Matrix.txt}. Считайте содержимое файла \monobf{Matrix.txt} в новый массив \monobf{y[n][n]} и выведите его на экран дисплея.
    \item Напишите программу, которая считывала бы элементы главной диагонали матрицы из файла \monobf{Matrix.txt}.
    \item Напишите программу, которая удаляла бы $k$-столбец ($1<k<M$) в файле \monobf{Matrix.txt}.
    \item Напишите программу, которая считывала бы элементы матрицы из файла \monobf{Matrix.txt} и записывала бы их в массив, соответствующего размера. Отсортируйте все столбцы матрицы по убыванию. Полученный массив запишите в файл \monobf{Matrix\_Sort.txt}.
    \item Дан текстовый файл, содержащий целые числа. Удалить из него все четные числа. 
    \item В данном текстовом файле удалить все слова, которые содержат хотя бы одну цифру. 
    \item Напишите программу, которая считывала бы саму себя и выводила бы на экран дисплея исходный текст программы в обратном порядке.
    \item Имеется файл с текстом. Осуществить шифрование данного текста в новый файл. Осуществить расшифровку полученного текста.
\end{enumerate}

\section{Структуры}
\subsection{Лабораторная работа 1}
\begin{enumerate}[leftmargin=*]
    \item Описать структуру с именем \monobf{AEROFLOT}, содержащую следующие поля:
        \begin{itemize}
            \item название пункта назначения рейса;
            \item номер рейса;
            \item тип самолета.
        \end{itemize}
    \item Написать программу, выполняющую следующие действия:
        \begin{itemize}
            \item ввод с клавиатуры данных в массив, состоящий из семи элементов типа \monobf{AEROFLOT}; записи должны быть размещены в алфавитном порядке по названиям пунктов назначения (для этого выполните процедуру сортировки);
            \item вывод на экран пунктов назначения и номеров рейсов, обслуживаемых самолетом, тип которого введен с клавиатуры. Если таких рейсов нет, выдать на дисплей соответствующее сообщение.
        \end{itemize}
    \item Описать структуру с именем \monobf{STUDENT}, содержащую следующие поля:
        \begin{itemize}
            \item \monobf{NAME} – фамилия и инициалы;
            \item \monobf{GROUP} – номер группы;
            \item \monobf{SES} - успеваемость (массив из пяти элементов).
        \end{itemize}
    \item Написать программу, выполняющую следующие действия:
        \begin{itemize}
            \item ввод с клавиатуры данных в массив \monobf{STUD1}, состоящий из десяти структур типа \monobf{STUDENT}; записи должны быть упорядочены по возрастанию содержимого поля \monobf{GROUP};
            \item вывод на дисплей фамилий и номеров групп для всех студентов, включенных в массив, если средний балл студента больше 4,0. Если таких нет, вывести соответствующее сообщение.
        \end{itemize}
\end{enumerate}
\subsection{Лабораторная работа 2}
\begin{enumerate}[leftmargin=*]
    \item Информация об итогах сдачи сессии каждым студентом представлена в следующем порядке: Фамилия Имя Отчество, номер группы, экзаменационные оценки по четырем предметам. 
    Отсортируйте фамилии студентов по алфавиту. Определить процент студентов, сдавших экзамены на 4 и 5.
    \item Ведомость успеваемости студентов курса содержит следующую информацию: номер группы, фамилию, средний балл за последнюю сессию. Составить список студентов в порядке возрастания их номеров групп.
    \item Даны два отсчета времени в часах, минутах и секундах. Найти величину временного интервала в секундах. Код реализовать через составной тип данных.
    \item Дано пять различных дат в виде: число, месяц, год. Вывести их на экран в порядке возрастания.
    \item Создать массив структур для учета занятости аудитории: день недели, время учебной пары, аудитория, название предмета. Реализовать поиск периодов времени, когда выбранная аудитория свободна.
    \item Список книг содержит следующую информацию: фамилии авторов, название книги, год издания. Найти все книги, в названии которых имеется определенное слово, например, "физика".
    \item Список имеющихся в продаже автомобилей содержит следующие сведения: марка автомобиля, цвет, стоимость, мощность двигателя, расход бензина на 100 км. Вывести перечень автомобилей, удовлетворяющих определенным требованиям клиента, таким например, как стоимость в диапазоне 300-500 тыс.руб., расход бензина в пределах 8-10 л и т.п.
    \item Описать два комплексных числа и проделать над ними операции сложения, вычитания, умножения и деления.
\end{enumerate}

\section{Обработка строк}
\subsection{Лабораторная работа 1}
\begin{enumerate}[leftmargin=*]
    \item В заданном тексте заменить все символы <<+>> на << - >>. В данной задаче воспользуйтесь массивом символов (Заголовочный файл \monobf{\color{LimeGreen}cstring}).
    \begin{lstlisting}
        #include<iostream>
        #include<cstring>
        using namespace std;
        int main(')
        {   char str[50]="(5+3)*7+65+7896";
            cout << "Initial string is: " << str << endl;
            for (int i=0; i<strlen(str); i++')
                if (str[i] == '+'')
                    str[i] = '-';
            cout << "Final string is: " << str << endl;
            return 0;
        }
    \end{lstlisting}
    \item В данном тексте посчитать число символов <<+>> и <<->>.
    \item Напишите программу, которая вычисляет длину введенной с клавиатуры строки. Реализуйте код программы, используя строковый тип данных (Заголовочный файл \monobf{\color{LimeGreen}string}').
    \begin{lstlisting}
        #include <iostream>
        #include <string>
        using namespace std;
        int main(')
        {
            setlocale(LC_ALL,"Russian"');
            string s;
            cout <<" %\color{Blue}Введите строку%: \n"; cin >> s;
            cout <<" %\color{Blue}Строка% " << s << " @\color{Blue}содержит@ "
                << s.length(') << " %\color{Blue}символ(а')%.\n";
            return 0;
        }
    \end{lstlisting}
    \item Задана строка символов. Определить, есть ли заданный символ «э» в этой строке символов. Выведите на экран номер первого вхождения данного символа в строке..
    \begin{lstlisting}
        #include <iostream>
        #include <string>
        using namespace std;
        int main(')
        {
            setlocale(LC_ALL,"Russian"');
            string str("@\color{Blue}Выведите на экран номер символа в строке@."');
            string symbol="@\color{Blue}э@";
            unsigned int pos str=str.find(symbol');
            if ((pos>=0') && pos(<str.length(')')')
                    cout << "@\color{Blue}Номер первого вхождения символа@ ("
                         << symbol << "') @\color{Blue}в строке@ \n\n("
                         << str << "')\n\n @\color{Blue}равен@: "<< pos << endl;
            else cout << "@\color{Blue}Такого символа в строке нет@!\n";
            return 0;
        }
    \end{lstlisting}
    \item Пусть задан некоторый текст. Вычислить, сколько раз повторяется наперед заданный символ <<\textbf{a}>>.
    \item Задана некоторая строка символов. Создать новую строку, которая образована из данной строки чтением от конца до начала.
    \item Задано слово. Проверить, читается ли это слово слева направо и наоборот. \textit{Простейшие \textbf{слова-палиндромы}}: мим, дед, наган, заказ, кабак, казак, мадам, шалаш.
    \item Вводится строка слов, разделенных пробелами. Найти самое длинное слово и вывести его на экран. 
    \item Пусть имеется текстовый файл, содержащий несколько строк символов. Подсчитать число символов <<->> в этих строках.
    \item Задана строка символов. Подсчитать число слов в этой строке. Считать, что слова разделяются одним из символов << >> (пробел), <<,>> (запятая), <<.>> (точка).
    \item Пусть имеется текстовый файл, содержащий несколько предложений. Подсчитать количество предложений и слов в этом файле.
    \item Пусть имеется текстовый файл, содержащий несколько слов. Отсортировать эти слова в алфавитном порядке и записать их в другой текстовый файл.
    \item Написать программу замены данных в строке. Пусть:\\ \textcolor{Red}{A = ''123456789''};\\
    \textcolor{Blue}{B = ''67''};\\
    \textcolor{Green}{C = ''-Шестьдесят семь-''};\\
    Необходимо найти символы \textbf{"67"} (из строки B) и заменить их на \textbf{"-Шестьдесят семь-"} (из строки C) в строке A, где A в итоге должна содержать \textbf{"12345-Шестьдесят семь-89"}.
\end{enumerate}
\subsection{Функции работы со строками}
\begin{table}[H]
    \centering
    \renewcommand{\arraystretch}{2}
    \begin{tabular}{|m{0.31\textwidth}|m{0.63\textwidth}|}
    \hline
    \multicolumn{1}{|c|}{Методы класса \textcolor{Green}{\monobf{String}}} & \multicolumn{1}{c|}{Описание метода}\\
    \hline
    \monobf{s.length()} & Возвращает длину строки \texttt{s} \\
    \hline
    \monobf{s.substr(pos,length)} & возвращает подстроку из строки \texttt{s}, начиная с номера
    \monobf{pos} длиной \monobf{length} символов; \\
    \hline
    \monobf{s.empty()} & возвращает значение \texttt{true}, если строка \texttt{s} пуста, \texttt{false} — в противном случае; \\
    \hline
    \monobf{s.insert(pos, s1)} & вставляет строку \monobf{s1} в строку \texttt{s}, начиная с позиции \texttt{pos}; \\
    \hline
    \monobf{s.remove(pos,length)} & удаляет из строки \texttt{s} подстроку \monobf{length} длинной \monobf{pos} символов; \\
    \hline
    \monobf{s.find(s1, pos)} & возвращает номер первого вхождения строки \monobf{s1} в строку \texttt{s}, поиск начинается с номера \monobf{pos}, параметр \monobf{pos} может отсутствовать, в этом случае поиск идет с начала строки; \\
    \hline
    \monobf{s.findfirst(s1, pos)} & возвращает номер первого вхождения любого символа из строки \monobf{s1} в строку \texttt{s}, поиск начинается с номера \monobf{pos}, который может отсутствовать.\\ 
    \hline
    \end{tabular}
    \caption{Функции работы со строками}
    \label{table_string_methods}
\end{table}

\section{Динамические типы данных}
\subsection{Лабораторная работа 1}
\begin{enumerate}[leftmargin=*]
    \item Напишите программу, реализующую объявление, заполнение и удаление динамического массива. Программа также должна выполнять вывод массива на экран и запись его в текстовый (бинарный) файл.
    \item Реализуйте предыдущую задачу с помощью подпрограмм (процедур и функций).
    \item Дана динамическая матрица случайных чисел размерности $N\times N$ ($N>9$). Вычислите произведение всех элементов матрицы, у которых индексы строк и столбцов четные. Результат выведите на экран.
    \item Описать структуру с именем \monobf{STUDENT}, содержащую следующие поля:
        \begin{itemize}
            \item \monobf{NAME} – фамилия и инициалы;
            \item \monobf{GROUP} – номер группы;
            \item \monobf{SES} - успеваемость (массив из пяти элементов).
        \end{itemize}
    Реализовать программу, используя указатели на структуру. Запишите данные для 10 студентов в файл.
    \item Создать структуру <<Товар>>. Каждый товар должен иметь не менее 8 полей, например, название; описание; страна и город, где произведен товар; предприятие-производитель; категория товара (продукты, хозтовары, промтовары и т.д.); цена; вес и т.д. Заполнить динамический массив десятью товарами. Реализовать поиск в массиве по названию, по вхождению слов в описание и по диапазону цены товара.
    \item Объявите указатель на массив типа \monobf{double} и предложите пользователю выбрать его размер. Далее напишите четыре функции: первая должна выделить память для массива, вторая – заполнить ячейки данными, третья – показать данные на экран, четвертая – освободить занимаемую память. Для обхода массива использовать указатели (запрещено обращаться к элементам массива по индексам).
\end{enumerate}
\begin{lstlisting}
    #include <iostream>
    #include <fstream>
    #include <cstdlib>

    using namespace std;
    int main(')
    {
        ofstream out("Array.txt"');
        const int n=10; double* x;
        x = new double[n];
        for (int i=0; i<n; i++')
        {
            *(x+i)=rand()%10;
            cout << "x["<< i << "]=" << *(x+i') << "\t";
            out << *(x+i) << "\n";
        }
        cout << "\n"; delete[] x; out.close(');

        ifstream in("Array.txt"');
        double * y;
        y = new double[n];
        for (int i=0; i<n; i++')
        {
            in >> *(y+i);
            cout << "y["<< i << "]="<< *(y+i') << "\t";
        }
        delete[] y; in.close('); return 0;
    }
\end{lstlisting}
\vspace{5cm}
\begin{lstlisting}
    #include <iostream>
    #include <fstream>
    #include <cstdlib>
    using namespace std;

    double* init(int n');
    void data(int n, double* x');
    void print(int n, double* x');
    void write_file(int n, double* x');
    void del(double* x');

    int main(')
    {
        int n; cout << "Input n: "; cin >> n;
        double* x=init(n');
        data(n,x');
        print(n,x');
        write_file(n,x');
        del(x');
        return 0;
    }
    double* init(int n') { return new double[n]; }
    void data(int n, double* x') { for (int i=0; i<n; i++') x[i]=rand(')%10; }
    void print(int n, double* x') { for (int i=0; i<n; i++') cout << x[i] << "\t"; }
    void write_file(int n, double* x')
    {
        ofstream out ("Array.txt"');
        for (int i=0; i<n; i++') out <<"x[" << i <<"]=" << x[i] << "\n";
        out.close(');
    }
    void del(double *x') { delete[] x; }
\end{lstlisting}
\vspace{5cm}
\begin{lstlisting}
    #include <iostream>
    using namespace std;
    int main(')
    {
        const int n=8, m=8;
        int **matrix;
        matrix=new int*[n];
        for (int i=0; i<n; i++')
            matrix[i]=new int[m];

        for (int i=0; i<n; i++')
        {
            for (int j=0; j<n; j++')
            {
                matrix[i][j]=(i+j); cout << matrix[i][j] << "\t";
            }
            cout << endl;
        }
        delete[] matrix;
        return 0;
    }
\end{lstlisting}


\section{ООП}
\subsection{Лабораторная работа 1}
\begin{enumerate}[leftmargin=*]
    \item Создать класс \monobf{Employee}. Класс должен включать помимо имени и фамилии, поле типа \monobf{int} для хранения номера сотрудника и поле типа \monobf{float} для хранения величины его оклада. Методы класса должны позволять пользователю вводить и отображать данные класса. Написать функцию \monobf{main()}, которая запросит пользователя ввести данные для трех сотрудников и выведет полученную информацию на экран.
    \item Создать класс типа круг. Поля-данные: радиус, координаты центра. Функции-члены вычисляют площадь, длину окружности, устанавливают поля и возвращают значения. Функции-члены установки полей класса должны проверять корректность задаваемых параметров (не равны нулю и не отрицательные).
    \item Создать класс типа время с полями: час (0–23), минуты (0–59), секунды (0–59). Класс имеет конструктор. Функции-члены установки времени, получения часа, минуты и секунды, а также две функции-члены печати: печать по шаблону «16 часов 18 минут 3 секунды» и «4 p.m. 18 минут 3 секунды». Функции-члены установки полей класса должны проверять корректность задаваемых параметров.
    \item Создать класс типа дата с полями: день (1–31), месяц (1–12), год (целое число). Класс имеет конструктор. Функции-члены установки дня, месяца и года, функции-члены получения дня, месяца и года, а также две функции-члены печати: печать по шаблону «5 января 1997 года» и «05.01.1997». Функции-члены установки полей класса должны проверять корректность задаваемых параметров.
    \item Создать класс одномерный массив целых чисел (вектор) с полями --- количество фактических элементов, массив (динамический). Функции-члены: обращения к отдельному элементу массива, вывода массива на экран, поэлементного сложения и вычитания со скаляром, вывода элемента по заданному индексу.
    \item Создать класс множество \monobf{Set}. Функции-члены реализуют добавление и удаление элемента, пересечение и разность множеств.
\end{enumerate}

\subsection{Лабораторная работа 2}
\begin{enumerate}[leftmargin=*]
    \item Создайте класс \monobf{Number}. Добавьте внутри класса функцию (метод) ввода переменной с клавиатуры и функцию вывода данной переменной на экран. Организуйте конструктор и деструктор с соответствующим выводом на экран сообщений \textbf{<<Сработал конструктор!>>} и <<Сработал деструктор!>>.
    \item Измените предыдущую программу таким образом, чтобы класс \monobf{Number} состоял из двух полей: целочисленной и символьной переменных. Организуйте работу деструктора и 4 конструкторов: конструктор без параметров, конструктор с целочисленным параметром, конструктор с символьным параметром, конструктор с обоими параметрами. 
    \item Создайте класс \monobf{Children}, который содержит такие поля (члены класса): закрытые (\monobf{private}) – имя, отчество и фамилию ребенка, а также его возраст; публичные (\monobf{public}) – методы ввода данных и отображения их на экран. Объявить два объекта класса, внести данные и показать их. Организуйте конструктор и деструктор с соответствующим выводом на экран сообщений \textbf{<<Сработал конструктор!>>} и \textbf{<<Сработал деструктор!>>}.
    \item Создать класс типа параллелепипед. Поля – высота, длина и ширина. Функции-члены вычисляют площадь и объем, сумму длин всех ребер параллелепипеда и длину главной диагонали, устанавливают поля и возвращают значения. Функции-члены установки полей класса должны проверять корректность задаваемых параметров (не равны нулю и не отрицательные). Организуйте два вида конструктора: без параметров и с параметрами по умолчанию, а также деструктор с сообщением об уничтожении объекта.
    \item Создайте класс <<Книга>>, содержащий следующие поля: название, количество страниц, год издания, цена. Методы: вычисления средней стоимости страницы; сколько лет книги; определение количества дней, прошедших после года издания книги. Создайте для данного класса конструктор и деструктор.
    \item Создайте класс одномерный динамический массив \monobf{Array}, который содержит такие поля (члены класса): публичные – методы ввода данных и отображения их на экран, а также определение максимального элемента массива. Создайте для данного класса конструктор и деструктор.
    \item Создайте класс \monobf{Matrix}, который содержит такие поля (члены класса): публичные – методы ввода данных и отображения их на экран, а также определение максимального элемента матрицы. Создайте для данного класса конструктор и деструктор.
\end{enumerate}