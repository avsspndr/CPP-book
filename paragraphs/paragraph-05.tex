\chapter{Динамические типы данных}
\section{Лабораторная работа 1}
\begin{enumerate}[leftmargin=*]
    \item Напишите программу, реализующую объявление, заполнение и удаление динамического массива. Программа также должна выполнять вывод массива на экран и запись его в текстовый (бинарный) файл.
    \item Реализуйте предыдущую задачу с помощью подпрограмм (процедур и функций).
    \item Дана динамическая матрица случайных чисел размерности $N\times N$ ($N>9$). Вычислите произведение всех элементов матрицы, у которых индексы строк и столбцов четные. Результат выведите на экран.
    \item Описать структуру с именем \monobf{STUDENT}, содержащую следующие поля:
        \begin{itemize}
            \item \monobf{NAME} – фамилия и инициалы;
            \item \monobf{GROUP} – номер группы;
            \item \monobf{SES} - успеваемость (массив из пяти элементов).
        \end{itemize}
    Реализовать программу, используя указатели на структуру. Запишите данные для 10 студентов в файл.
    \item Создать структуру <<Товар>>. Каждый товар должен иметь не менее 8 полей, например, название; описание; страна и город, где произведен товар; предприятие-производитель; категория товара (продукты, хозтовары, промтовары и т.д.); цена; вес и т.д. Заполнить динамический массив десятью товарами. Реализовать поиск в массиве по названию, по вхождению слов в описание и по диапазону цены товара.
    \item Объявите указатель на массив типа \monobf{double} и предложите пользователю выбрать его размер. Далее напишите четыре функции: первая должна выделить память для массива, вторая – заполнить ячейки данными, третья – показать данные на экран, четвертая – освободить занимаемую память. Для обхода массива использовать указатели (запрещено обращаться к элементам массива по индексам).
\end{enumerate}
\begin{lstlisting}
    #include <iostream>
    #include <fstream>
    #include <cstdlib>

    using namespace std;
    int main(')
    {
        ofstream out("Array.txt"');
        const int n=10; double* x;
        x = new double[n];
        for (int i=0; i<n; i++')
        {
            *(x+i)=rand()%10;
            cout << "x["<< i << "]=" << *(x+i') << "\t";
            out << *(x+i) << "\n";
        }
        cout << "\n"; delete[] x; out.close(');

        ifstream in("Array.txt"');
        double * y;
        y = new double[n];
        for (int i=0; i<n; i++')
        {
            in >> *(y+i);
            cout << "y["<< i << "]="<< *(y+i') << "\t";
        }
        delete[] y; in.close('); return 0;
    }
\end{lstlisting}
\vspace{5cm}
\begin{lstlisting}
    #include <iostream>
    #include <fstream>
    #include <cstdlib>
    using namespace std;

    double* init(int n');
    void data(int n, double* x');
    void print(int n, double* x');
    void write_file(int n, double* x');
    void del(double* x');

    int main(')
    {
        int n; cout << "Input n: "; cin >> n;
        double* x=init(n');
        data(n,x');
        print(n,x');
        write_file(n,x');
        del(x');
        return 0;
    }
    double* init(int n') { return new double[n]; }
    void data(int n, double* x') { for (int i=0; i<n; i++') x[i]=rand(')%10; }
    void print(int n, double* x') { for (int i=0; i<n; i++') cout << x[i] << "\t"; }
    void write_file(int n, double* x')
    {
        ofstream out ("Array.txt"');
        for (int i=0; i<n; i++') out <<"x[" << i <<"]=" << x[i] << "\n";
        out.close(');
    }
    void del(double *x') { delete[] x; }
\end{lstlisting}
\vspace{5cm}
\begin{lstlisting}
    #include <iostream>
    using namespace std;
    int main(')
    {
        const int n=8, m=8;
        int **matrix;
        matrix=new int*[n];
        for (int i=0; i<n; i++')
            matrix[i]=new int[m];

        for (int i=0; i<n; i++')
        {
            for (int j=0; j<n; j++')
            {
                matrix[i][j]=(i+j'); cout << matrix[i][j] << "\t";
            }
            cout << endl;
        }
        delete[] matrix;
        return 0;
    }
\end{lstlisting}