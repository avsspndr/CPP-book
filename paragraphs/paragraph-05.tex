\chapter{Динамические типы данных}
\begin{enumerate}[leftmargin=*]
    \item Напишите программу, реализующую объявление, заполнение и удаление динамического массива. Программа также должна выполнять вывод массива на экран и запись его в текстовый (бинарный) файл.
    \item Реализуйте предыдущую задачу с помощью подпрограмм (процедур и функций).
    \item Дана динамическая матрица случайных чисел размерности $N\times N$ ($N>9$). Вычислите произведение всех элементов матрицы, у которых индексы строк и столбцов четные. Результат выведите на экран.
    \item Описать структуру с именем \monobf{STUDENT}, содержащую следующие поля:
        \begin{itemize}
            \item \monobf{NAME} – фамилия и инициалы;
            \item \monobf{GROUP} – номер группы;
            \item \monobf{SES} - успеваемость (массив из пяти элементов).
        \end{itemize}
    Реализовать программу, используя указатели на структуру. Запишите данные для 10 студентов в файл.
    \item Создать структуру <<Товар>>. Каждый товар должен иметь не менее 8 полей, например, название; описание; страна и город, где произведен товар; предприятие-производитель; категория товара (продукты, хозтовары, промтовары и т.д.); цена; вес и т.д. Заполнить динамический массив десятью товарами. Реализовать поиск в массиве по названию, по вхождению слов в описание и по диапазону цены товара.
    \item Объявите указатель на массив типа \monobf{double} и предложите пользователю выбрать его размер. Далее напишите четыре функции: первая должна выделить память для массива, вторая – заполнить ячейки данными, третья – показать данные на экран, четвертая – освободить занимаемую память. Для обхода массива использовать указатели (запрещено обращаться к элементам массива по индексам).
\end{enumerate}
%\begin{lstlisting}
%    #include <iostream>
%    using namespace std;
%    int main(')
%    {
%        const int n=8, m=8;
%        int **matrix;
%        matrix=new int*[n];
%        for (int i=0; i<n; i++')
%            matrix[i]=new int[m];
%
%        for (int i=0; i<n; i++')
%        {
%            for (int j=0; j<n; j++')
%           {
%                matrix[i][j]=(i+j'); cout << matrix[i][j] << "\t";
%            }
%            cout << endl;
%        }
%        delete[] matrix;
%        return 0;
%    }
%\end{lstlisting}