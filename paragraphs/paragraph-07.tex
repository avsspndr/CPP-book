\chapter{Функции и указатели}
\section{Лабораторная работа 1. Указатели}
\begin{enumerate}[leftmargin=*]
    \item С одномерным массивом, состоящим из $n$ вещественных элементов, выполнить следующее: Преобразовать массив таким образом, чтобы сначала располагались все положительные элементы, а потом – все отрицательные (элементы, равные 0, считать положительными).
    \item С одномерным массивом, состоящим из $n$ вещественных элементов, выполнить следующее: Преобразовать массив таким образом, чтобы сначала располагались все элементы, целая часть которых лежит в интервале $[a,b]$, а потом – все остальные.
    \item С одномерным массивом, состоящим из n вещественных элементов, выполнить следующее: Преобразовать массив таким образом, чтобы сначала располагались все отрицательные элементы, а потом – все положительные (элементы, равные 0, считать положительными).
    \item С одномерным массивом, состоящим из $n$ вещественных элементов, выполнить следующее: Преобразовать массив таким образом, чтобы сначала располагались все элементы, целая часть которых не превышает 1, а потом – все остальные.
    \item С одномерным массивом, состоящим из $n$ вещественных элементов, выполнить следующее: Преобразовать массив таким образом, чтобы сначала располагались все элементы, отличающиеся от максимального не более чем на 20\%, а потом – все остальные.
    \item С одномерным массивом, состоящим из $n$ вещественных элементов, выполнить следующее: Заменить все отрицательные элементы массива их модулями и изменить порядок следования элементов в массиве на обратный.
    \item С одномерным массивом, состоящим из $n$ вещественных элементов, выполнить следующее: Сжать массив, удалив из него одинаковые элементы.  Освободившиеся в конце массива элементы заполнить нулями.
    \item С одномерным массивом, состоящим из $n$ вещественных элементов, выполнить следующее: Сжать массив, удалив из него все элементы, модуль которых не превышает 1.  Освободившиеся в конце массива элементы заполнить нулями.
    \item С одномерным массивом, состоящим из $n$ вещественных элементов, выполнить следующее: Сжать массив, удалив из него все элементы, модуль которых находится в интервале $[a,b]$.  Освободившиеся в конце массива элементы заполнить нулями.
    \item С одномерным массивом, состоящим из $n$ вещественных элементов, выполнить следующее: Преобразовать массив таким образом, чтобы сначала располагались все элементы, равные нулю, а потом – все остальные.
    \item С одномерным массивом, состоящим из $n$ вещественных элементов, выполнить следующее: Преобразовать массив таким образом, чтобы в первой его половине располагались элементы,  стоявшие в нечетных позициях, а во второй половине – элементы, стоявшие в четных позициях.
    \item С одномерным массивом, состоящим из $n$ вещественных элементов, выполнить следующее: Преобразовать массив таким образом, чтобы сначала располагались все элементы, модуль которых не превышает 1, а потом – все остальные.
    \item С одномерным массивом, состоящим из $n$ вещественных элементов, выполнить следующее: Преобразовать массив таким образом, чтобы элементы, равные нулю, располагались после всех остальных.
    \item С одномерным массивом, состоящим из $n$ вещественных элементов, выполнить следующее: Преобразовать массив таким образом, чтобы в первой его половине располагались элементы,  стоявшие в четных позициях, а во второй половине – элементы, стоявшие в нечетных позициях.
    \item С одномерным массивом, состоящим из $n$ вещественных элементов, выполнить следующее: Сжать массив, удалив из него все элементы, величина которых находится в интервале $[a,b]$. Освободившиеся в конце массива элементы заполнить нулями.
\end{enumerate}
\section{Лабораторная работа 2. Подпрограммы-функции}
\begin{enumerate}[leftmargin=*]
    \item Напишите функцию, которая возвращает большее значение из введенных пользователем.
    \item Напишите программу, содержащую функцию, которая возводит число $a$ в степень $b$. Причем $a$ и $b$ вводятся с клавиатуры.
    \item Напишите функцию, вычисляющую процент от числа. Например: \textit{321\% от числа 3 равен 9.63}.
    \item Сделайте программу, функция которой сравнивает введенные числа и результат выдает в виде знаков ">", "<" или "=".
    \item Написать функцию, вычисляющую корни квадратного уравнения. В качестве аргументов она принимает коэффициенты $(a, b, c)$, а возвращает значение по обстоятельству ($x_1$ и $x_2$, либо «Корней нет», либо $а=0$ «Введены не корректные данные»).
    \item Напишите функцию, которая возвращает 1, если пользователь ввел гласную букву латинского алфавита, и 0 в противном случае.
    \item Написать функцию, специализированную на вывод строки из звездочек, количество которых определяется пользователем.
    \item Написать и протестировать функцию, которая из заданного массива формирует новый массив, состоящий только из элементов, дважды входящих в первый массив.
    \item Написать и протестировать функцию, возвращающую номер самого последнего элемента из массива, который совпадает с заданным с клавиатуры числом. Если такого элемента нет, функция должна возвращать "–1".
\end{enumerate}
\section{Лабораторная работа 3. Функции, указатели}
\begin{enumerate}[leftmargin=*]
    \item Создайте программу, реализующую работу с динамическим массивом. Разработайте 4 функции: первая - инициализация массива с выделением памяти под массив, вторая – заполнение массива данными, третья – вывод данных на экран, четвертая – освобождение занимаемой массивом памяти. 
    \item В целочисленном динамическом массиве \monobf{x[20]} определить сумму положительных элементов, делящихся на 5 без остатка и поставить ее на место максимального элемента массива \monobf{y[10]}. Реализуйте в виде отдельных функций: 1) создание массивов; 2) поиск элементов массива \monobf{x[20]}; 3) замена соответствующего элемента массива \monobf{y[10]}; 4) освобождение занимаемой массивами памяти.
    \item Программа. Описать функцию $f(x, n, p)$, определяющую, чередуются ли положительные и отрицательные элементы в целочисленном динамическом массиве \monobf{x[n]} из $n$ элементов и вычисляющую целочисленное значение $p$. Если элементы чередуются, то $p$ – это сумма положительных элементов, иначе $p$ – это произведение отрицательных элементов. С помощью этой функции провести анализ целочисленного массива \monobf{x[50]}.
    \item Создайте динамический массив случайных чисел. Перемешать его элементы случайным образом так, чтобы каждый элемент оказался на новом месте. Реализуйте программу в виде отдельных подпрограмм-функций.
    \item Дан динамический массив символов. Показать номера символов, совпадающих с последним символом строки. Реализуйте программу в виде отдельных подпрограмм-функций.
\end{enumerate}

\section{Самостоятельная работа 1. Функции и файлы}
\begin{enumerate}[leftmargin=*]
    \item Напишите функцию поиска в массиве количества чисел, соседи у которых отличаются более чем в 2 раза. Реализуйте в программе считывание данных (массива) из файла.
    \item Напишите подпрограмму-функцию, определяющую образует ли элементы массива в данном порядке арифметическую или геометрическую прогрессии.
    \item Напишите функцию поиска в массиве максимального количества одинаковых элементов. Выведите на экран значение этого элемента, их количество и номера в массиве.
    \item Написать функцию, специализированную на вывод строки из звездочек, количество которых определяется пользователем.
    Написать функцию, вычисляющую биномиальный коэффициент $C_n^k$. Результаты вычислений запишите в файл.
    \item Написать и протестировать функцию, которая из заданного массива формирует новый массив, состоящий только из элементов, дважды входящих в первый массив. Реализуйте в программе считывание и запись данных (массива) в файл.
    \item Напишите программу, которая считывала бы саму себя и выводила бы на экран дисплея исходный текст программы в обратном порядке.
    \item Имеется файл с текстом. Осуществить шифрование данного текста в новый файл. Осуществить расшифровку полученного текста.
\end{enumerate}

\section{Самостоятельная работа 2. Функции, указатели}
\begin{enumerate}[leftmargin=*]
    \item Создайте программу, реализующую работу с динамическим массивом. Разработайте 5 функций: первая - инициализация массива с выделением памяти под массив; вторая – заполнение массива случайными числами; третья – вывод данных на экран; четвертая – сортировка массива, таким образом, чтобы первая половина массива была отсортирована по возрастанию, а вторая по убыванию; пятая – освобождение занимаемой массивом памяти.
    \item Напишите функцию поиска в динамическом массиве количества чисел, соседи у которых отличаются более чем в 2 раза.
    \item Напишите подпрограмму-функцию, определяющую образует ли элементы динамического массива в данном порядке арифметическую или геометрическую прогрессии.
    \item Напишите функцию поиска в динамическом массиве максимального количества одинаковых элементов. Выведите на экран значение этого элемента, их количество и номера в массиве.
    \item Напишите функцию, осуществляющую циклический сдвиг динамического массива на $k$ единиц вправо, если первый наименьший элемент массива расположен раньше последнего наибольшего элемента массива, и влево, если иначе.
    \item В данном динамическом массиве каждый элемент равен 0, 1  или 2. Переставить элементы массива так, чтобы сначала располагались все нули, затем все единицы и, наконец, все двойки. Дополнительный массив не использовать. Реализуйте алгоритм в виде отдельной подпрограммы-функции
\end{enumerate}