\chapter{Общие задачи}
\begin{enumerate}[leftmargin=*]
    \item Вычисление числа $\pi$. Для вычисления числа π используем ряд
    \begin{equation*}
        \frac{\pi}{4}=1-\frac{1}{3}+\frac{1}{5}-\frac{1}{7}+\frac{1}{9}-\dots .
    \end{equation*}
    Провести вычисления, обеспечив заранее заданную точность $\varepsilon>0$. При этом вычисления заканчиваются при $a < \varepsilon$.
    \item Напишите программу, которая находит факториал от введенного числа. Реализуйте алгоритм в виде рекурсивной функции.
    \item Определить число $e$ --- основание натуральных логарифмов с помощью ряда:
    \begin{equation*}
        e=1+\frac{1}{1!}+\frac{1}{2!}+\frac{1}{3!}+\frac{1}{4!}+\dots+\frac{1}{n!} .
    \end{equation*}
    Вычислить $е$ для всех значений $n$ от 1 до 20. Для каждого случая вывести на экран $n$ и соответствующее значение $е$.
    \item Дан текстовый файл $f$, компоненты которого являются целыми числами. Записать в файл $g$ все четные числа файла $f$, а в файл $h$ – все нечетные. Порядок следования чисел сохраняется.
    \item Ввести натуральное $N$ и проверить, является ли оно совершенным? Примечание: совершенное число равно сумме всех своих делителей, исключая само число. Например, 6 = 1 + 2 + 3.
    \item Составить программу для нахождения всех автоморфных чисел в отрезке $[m, n]$. Автоморфным называется целое число, которое равно последним числам своего квадрата. 
    \textit{Например:} 52 = 25, 62 = 36, 252 = 625.
    \item Известно, что сумма $N$ первых нечетных чисел равна квадрату числа $N$. Например, $1+3+5=3^2$, $1+3+5+7=4^2$ и т.д. Ввести натуральное К и распечатать таблицу всех натуральных чисел от $1$ до $К$ и их квадратов с использованием указанного соотношения.
    \item Дано натуральное число $n$ $(n≤100)$, определяющее возраст человека (в годах). Дать для этого числа наименования «год», «года», или «лет». Например, 1 год, 23 года, 45 лет и т.д.
    \item Написать программу вычисления методом Монте-Карло площади фигуры, ограниченной половиной синусоиды.
    \item Задача на перебор. Получить все перестановки элементов $1, … , 6$.
    \item Написать программу для вычисления методом Монте-Карло площади $S$ тела, ограниченного кривыми $xy=a$ и $x+y=\frac{5}{2}a$. Сравнить результат с точным значением.
    \item Игра «Угадай число». Один из играющих задумывает число от 1 до 1000, другой пытается угадать его за десять вопросов вида: верно ли, что задуманное число больше такого-то числа. Написать программу, играющую за отгадчика.
    \item Пусть даны четыре целых числа \textit{(hour, min, sec, time)}. Первые три из них \textit{(hour, min, sec)} – это время запуска ракеты в часах, минутах и секундах. Четвертое \textit{(time)} –-- определяет время полета в секундах. Вычислить время возвращения ракеты на землю.
    \item Один из простейших способов шифровки текста состоит в табличной замене каждого символа другим символом --- его шифром. Выбрать некоторую таблицу, разработать способ ее представления, затем
    \begin{enumerate}[label=\asbuk*)]
        \item зашифровать данный текст;
        \item расшифровать данный текст.
    \end{enumerate}
    \item Численно решить уравнение радиоактивного распада:
    \begin{equation*}
        \dfrac{dN}{dt} = -\lambda N .
    \end{equation*}
    Разработать алгоритм решения  задачи и написать программу на языке программирования C\texttt{++}. Сравнить численное решение с аналитическим. Определить условия сходимости.
    \item Создать типизированный файл записей со сведениями о телефонах абонентов; каждая запись имеет поля: фамилия абонента, год установки телефона, номер телефона. По заданной фамилии абонента выдать номера его телефонов. Определить количество установленных телефонов с $N$-го года. Отсортировать список по алфавиту и вывести все записи на экран.
    \item Напишите функцию, которая преобразовывает значение, заданное в радианах, в значение, выраженное в градусах, угловых минутах и угловых секундах. Воспользуйтесь указателем на структурный тип данных.
    \item Напишите программу, которая считывает числовые значения из файла, вычисляет значение полусуммы наибольшего и наименьшего элементов, а затем подсчитывает количество значений, не превышающих по величине полусумму, и больших чем полусумма.
    \item Напишите программу, которая меняет местами столбцы матрицы, содержащие наибольший и наименьший элементы.
    \item Треугольник задан координатами трех своих вершин. Определить, где находится точка $О$ с указанными координатами - внутри или вне треугольника.
    \item Составьте алгоритм и напишите программу для вычисления приближенного значения натурального логарифма от произвольного значения аргумента $|x|<1$, вводимого с клавиатуры. Ряд Тейлора для этой функции имеет вид:
    \begin{equation*}
        ln(1+x)=x-\frac{x^2}{2!}+\frac{x^3}{3}-{x^4}{4}+\dots .
    \end{equation*}
    \item Напишите программу, которая "сжимает" текстовый файл, считывая его элементы и заменяя все повторяющиеся символы, например, \textbf{ccccc}.... текстом \textbf{c(n)}, где \textbf{n}-число повторений символа \textbf{c}. В программе используйте процедуры-функции.
\end{enumerate}