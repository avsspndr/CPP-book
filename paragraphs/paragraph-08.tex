\chapter{ООП}
\section{Лабораторная работа 1}
\begin{enumerate}[leftmargin=*]
    \item Создать класс \monobf{Employee}. Класс должен включать помимо имени и фамилии, поле типа \monobf{int} для хранения номера сотрудника и поле типа \monobf{float} для хранения величины его оклада. Методы класса должны позволять пользователю вводить и отображать данные класса. Написать функцию \monobf{main()}, которая запросит пользователя ввести данные для трех сотрудников и выведет полученную информацию на экран.
    \item Создать класс типа круг. Поля-данные: радиус, координаты центра. Функции-члены вычисляют площадь, длину окружности, устанавливают поля и возвращают значения. Функции-члены установки полей класса должны проверять корректность задаваемых параметров (не равны нулю и не отрицательные).
    \item Создать класс типа время с полями: час (0–23), минуты (0–59), секунды (0–59). Класс имеет конструктор. Функции-члены установки времени, получения часа, минуты и секунды, а также две функции-члены печати: печать по шаблону «16 часов 18 минут 3 секунды» и «4 p.m. 18 минут 3 секунды». Функции-члены установки полей класса должны проверять корректность задаваемых параметров.
    \item Создать класс типа дата с полями: день (1–31), месяц (1–12), год (целое число). Класс имеет конструктор. Функции-члены установки дня, месяца и года, функции-члены получения дня, месяца и года, а также две функции-члены печати: печать по шаблону «5 января 1997 года» и «05.01.1997». Функции-члены установки полей класса должны проверять корректность задаваемых параметров.
    \item Создать класс одномерный массив целых чисел (вектор) с полями --- количество фактических элементов, массив (динамический). Функции-члены: обращения к отдельному элементу массива, вывода массива на экран, поэлементного сложения и вычитания со скаляром, вывода элемента по заданному индексу.
    \item Создать класс множество \monobf{Set}. Функции-члены реализуют добавление и удаление элемента, пересечение и разность множеств.
\end{enumerate}

section{Лабораторная работа 2}
\begin{enumerate}[leftmargin=*]
    \item Создайте класс \monobf{Number}. Добавьте внутри класса функцию (метод) ввода переменной с клавиатуры и функцию вывода данной переменной на экран. Организуйте конструктор и деструктор с соответствующим выводом на экран сообщений \textbf{<<Сработал конструктор!>>} и <<Сработал деструктор!>>.
    \item Измените предыдущую программу таким образом, чтобы класс \monobf{Number} состоял из двух полей: целочисленной и символьной переменных. Организуйте работу деструктора и 4 конструкторов: конструктор без параметров, конструктор с целочисленным параметром, конструктор с символьным параметром, конструктор с обоими параметрами. 
    \item Создайте класс \monobf{Children}, который содержит такие поля (члены класса): закрытые (\monobf{private}) – имя, отчество и фамилию ребенка, а также его возраст; публичные (\monobf{public}) – методы ввода данных и отображения их на экран. Объявить два объекта класса, внести данные и показать их. Организуйте конструктор и деструктор с соответствующим выводом на экран сообщений \textbf{<<Сработал конструктор!>>} и \textbf{<<Сработал деструктор!>>}.
    \item Создать класс типа параллелепипед. Поля – высота, длина и ширина. Функции-члены вычисляют площадь и объем, сумму длин всех ребер параллелепипеда и длину главной диагонали, устанавливают поля и возвращают значения. Функции-члены установки полей класса должны проверять корректность задаваемых параметров (не равны нулю и не отрицательные). Организуйте два вида конструктора: без параметров и с параметрами по умолчанию, а также деструктор с сообщением об уничтожении объекта.
    \item Создайте класс <<Книга>>, содержащий следующие поля: название, количество страниц, год издания, цена. Методы: вычисления средней стоимости страницы; сколько лет книги; определение количества дней, прошедших после года издания книги. Создайте для данного класса конструктор и деструктор.
    \item Создайте класс одномерный динамический массив \monobf{Array}, который содержит такие поля (члены класса): публичные – методы ввода данных и отображения их на экран, а также определение максимального элемента массива. Создайте для данного класса конструктор и деструктор.
    \item Создайте класс \monobf{Matrix}, который содержит такие поля (члены класса): публичные – методы ввода данных и отображения их на экран, а также определение максимального элемента матрицы. Создайте для данного класса конструктор и деструктор.
\end{enumerate}