\chapter{ООП}
\section{Лабораторная работа 1}
\begin{enumerate}[leftmargin=*]
    \item Создать класс \monobf{Employee}. Класс должен включать помимо имени и фамилии, поле типа \monobf{int} для хранения номера сотрудника и поле типа \monobf{float} для хранения величины его оклада. Методы класса должны позволять пользователю вводить и отображать данные класса. Написать функцию \monobf{main()}, которая запросит пользователя ввести данные для трех сотрудников и выведет полученную информацию на экран.
    \item Создать класс типа круг. Поля-данные: радиус, координаты центра. Функции-члены вычисляют площадь, длину окружности, устанавливают поля и возвращают значения. Функции-члены установки полей класса должны проверять корректность задаваемых параметров (не равны нулю и не отрицательные).
    \item Создать класс типа время с полями: час (0–23), минуты (0–59), секунды (0–59). Класс имеет конструктор. Функции-члены установки времени, получения часа, минуты и секунды, а также две функции-члены печати: печать по шаблону «16 часов 18 минут 3 секунды» и «4 p.m. 18 минут 3 секунды». Функции-члены установки полей класса должны проверять корректность задаваемых параметров.
    \item Создать класс типа дата с полями: день (1–31), месяц (1–12), год (целое число). Класс имеет конструктор. Функции-члены установки дня, месяца и года, функции-члены получения дня, месяца и года, а также две функции-члены печати: печать по шаблону «5 января 1997 года» и «05.01.1997». Функции-члены установки полей класса должны проверять корректность задаваемых параметров.
    \item Создать класс одномерный массив целых чисел (вектор) с полями --- количество фактических элементов, массив (динамический). Функции-члены: обращения к отдельному элементу массива, вывода массива на экран, поэлементного сложения и вычитания со скаляром, вывода элемента по заданному индексу.
    \item Создать класс множество \monobf{Set}. Функции-члены реализуют добавление и удаление элемента, пересечение и разность множеств.
\end{enumerate}

\section{Лабораторная работа 2}
\begin{enumerate}[leftmargin=*]
    \item Создайте класс \monobf{Number}. Добавьте внутри класса функцию (метод) ввода переменной с клавиатуры и функцию вывода данной переменной на экран. Организуйте конструктор и деструктор с соответствующим выводом на экран сообщений \textbf{<<Сработал конструктор!>>} и <<Сработал деструктор!>>.
    \begin{lstlisting}
        #include <iostream>
        using namespace std;
        class Number
        {
            int a;
          public:
             Number(')
             { cout << "@\textcolor{Blue}{Сработал конструктор без параметров}@" << "\n"; }
             Number(int A')
             {
                 a = A;
                 cout << "@\textcolor{Blue}{Сработал конструктор с параметром}@: " << "\n";
                 cout << "a= " << a << "\n";
             }
             void set_Number(')
             {  cout << "@\textcolor{Blue}{Введите целое число a= }@"; cin >> a; }
             void out_Number(')
             {  cout << "@\textcolor{Blue}{Число a= }@" << a << "\n"; }
             ~Number(')
             { cout << "\n@\textcolor{Blue}{Сработал деструктор}@" << "\n"; }
        };
        int main(')
        {
            setlocale(0, "rus"');
            cout << "\n@\textcolor{Blue}{***Первый объект***}@" << "\n";
            Number obj@1@;
            obj@1@.set_Number(');
            obj@1@.out_Number(');

            cout << "\n@\textcolor{Blue}{***Второй объект***}@" << "\n";
            Number obj@2@(100');
            obj@2@.out_Number(');
            return 0;
        }
    \end{lstlisting}
    \item Измените предыдущую программу таким образом, чтобы класс \monobf{Number} состоял из двух полей: целочисленной и символьной переменных. Организуйте работу деструктора и 4 конструкторов: конструктор без параметров, конструктор с целочисленным параметром, конструктор с символьным параметром, конструктор с обоими параметрами. 
    \item Создайте класс \monobf{Children}, который содержит такие поля (члены класса): закрытые (\monobf{private}) – имя, отчество и фамилию ребенка, а также его возраст; публичные (\monobf{public}) – методы ввода данных и отображения их на экран. Объявить два объекта класса, внести данные и показать их. Организуйте конструктор и деструктор с соответствующим выводом на экран сообщений \textbf{<<Сработал конструктор!>>} и \textbf{<<Сработал деструктор!>>}.
    \item Создать класс типа параллелепипед. Поля – высота, длина и ширина. Функции-члены вычисляют площадь и объем, сумму длин всех ребер параллелепипеда и длину главной диагонали, устанавливают поля и возвращают значения. Функции-члены установки полей класса должны проверять корректность задаваемых параметров (не равны нулю и не отрицательные). Организуйте два вида конструктора: без параметров и с параметрами по умолчанию, а также деструктор с сообщением об уничтожении объекта.
    \item Создайте класс <<Книга>>, содержащий следующие поля: название, количество страниц, год издания, цена. Методы: вычисления средней стоимости страницы; сколько лет книги; определение количества дней, прошедших после года издания книги. Создайте для данного класса конструктор и деструктор.
    \item Создайте класс одномерный динамический массив \monobf{Array}, который содержит такие поля (члены класса): публичные – методы ввода данных и отображения их на экран, а также определение максимального элемента массива. Создайте для данного класса конструктор и деструктор.
    \item Создайте класс \monobf{Matrix}, который содержит такие поля (члены класса): публичные – методы ввода данных и отображения их на экран, а также определение максимального элемента матрицы. Создайте для данного класса конструктор и деструктор.
\end{enumerate}

\section{Лабораторная работа 3}
\begin{enumerate}[leftmargin=*]
    \item Определить базовый класс \textbf{Автомобиль} с полями \textit{Торговая марка, Число цилиндров, Мощность}. Создать конструкторы и деструктор объектов, а также метод \monobf{Show()}, выводящий информацию об объекте. Определить производный класс Грузовик, добавив в него /textit{характеристику грузоподъемности кузова}. Создать конструкторы объектов производного класса. Переопределить метод \monobf{Show()} в производном классе. Создать методы, позволяющие изменять поля объектов базового и производного классов.
    \item Создайте класс с именем \textbf{CPerson}, содержащий три поля типа \monobf{string} для хранения имени, фамилии и отчества. В классе создайте функцию \monobf{ShowData()}, выводящую на экран имя, фамилию и отчество. Далее от класса \textbf{CPerson} с помощью наследования создайте два класса: \textbf{CStudent},\textbf{CProfessor}. К классу \textbf{CStudent} добавьте дополнительное поле, содержащее средний бал студента. К классу \textbf{CProfessor} три поля: 1) число публикаций профессора, 2) должность (тип - перечисление) - преподаватель, старший преподаватель, доцент, профессор, 3) возраст. Для каждого производного класса переопределите метод \monobf{ShowData()}. В основной программе определите массив (можно не динамический) указателей на объекты класса \textbf{CPerson}. Далее в цикле нужно организовать ввод студентов и профессоров вперемешку. Когда ввод будет закончен, нужно вывести информацию с помощью метода \monobf{ShowData()} обо всех людях.
\end{enumerate}

\section{Лабораторная работа}
\begin{enumerate}[leftmargin=*]
    \item Создайте два базовых класса \monobf{ClassX} и \monobf{ClassY}, содержащие такие поля (члены класса): защищенные (\monobf{protected}) – переменная вещественного типа (\monobf{x} для класса \monobf{ClassX} и \monobf{y} для класса \monobf{ClassY}); публичные (\monobf{public}) – методы ввода данных и отображения их на экран. Создайте производный класс \monobf{ClassZ},  содержащий публичный (\monobf{public}) метод – метод расчета произведения \monobf{x*y}.
    \begin{lstlisting}
        #include <iostream>
        using namespace std;
        class ClassX
        {
            protected:
                double x;
            public:
                void setX(')
                {cout << "Input x: "; cin >> x;}
                void outX(')
                {cout << "x= " << x << "\n";}
        };
        class ClassY
        {
            protected:
                double y;
            public:
                void SetY(')
                {cout << "Input y: "; cin >> y;}
                void outY(')
                {cout << "y= " << y << "\n";}
        };
        class ClassZ: public Class X, public Class Y
        {
            public:
                int make_xy(') { return x*y; }
        };
        int main(')
        {
            classZ obj;
            obj.setX('); obj.setY(');
            obj.outX('); obj.outY(');
            cout << "xy= " << obj.make_xy(') << "\n";
            return 0;
        }
    \end{lstlisting}
    \item В предыдущей задаче организуйте работу конструктора и деструктора.
    \item Создайте базовый класс \monobf{Human} и класс-наследник \monobf{Student}. Класс \monobf{Human} описывает модель человека. В нем хранятся имя и фамилия, дата рождения, адрес прописки. Конструктор \monobf{Student} принимает все аргументы конструктора базового класса, а также дополнительные аргументы для расширения функционала, такие как, список оценок студента по предметам. Класс \monobf{Student} содержит методы вычисления среднего балла студента и вывода его на экран.
    \item Разработать три класса, которые следует связать между собой, используя наследование: (1) класс \monobf{Product}, который имеет три элемент-данных — имя, цена и вес товара (базовый класс для всех классов); (2) класс  \monobf{Buy}, содержащий данные о количестве покупаемого товара в штуках, о цене за весь купленный товар и  о весе товара (производный класс для класса \monobf{Product} и базовый класс для класса \monobf{Check}; (3) класс \monobf{Check}, не содержащий никаких элемент-данных. Данный класс должен выводить на экран информацию о товаре и о покупке (производный класс для класса \monobf{Buy}); Для взаимодействия с данными классов разработать \monobf{set}- и \monobf{get}-методы. Все элемент-данные классов объявлять как \monobf{private}.
\end{enumerate}

\section{Самостоятельная работа}
\begin{enumerate}[leftmargin=*]
    \item Класс \textbf{Покупатель}: Фамилия, Имя, Отчество, Адрес, Номер кредитной карточки, Номер банковского счета; Конструктор; Методы: установка значений атрибутов, получение значений атрибутов, вывод информации. Создать массив объектов данного класса. Вывести список покупателей в алфавитном порядке и список покупателей, у которых номер кредитной карточки находится в заданном диапазоне.
    \item Создайте класс с именем \textbf{Train}, содержащий поля: название пункта назначения, номер поезда, время отправления. Ввести данные в массив из пяти элементов типа \textbf{Train}, упорядочить элементы по номерам поездов. Добавить возможность вывода информации о поезде, номер которого введен пользователем. Добавить возможность сортировки массив по пункту назначения, причем поезда с одинаковыми пунктами назначения должны быть упорядочены по времени отправления. 
\end{enumerate}

\section{Лабораторная работа 4. Шаблоны функций}
\begin{enumerate}[leftmargin=*]
    \item Написать программу для нахождения максимального значения в массиве из целочисленных, вещественных, символьных и строковых величин с использованием шаблона функции.
    \item Написать шаблонную функцию, которая примет два числа, определит максимальное из них и вернет его в программу. Будем иметь в виду, что в функцию мы можем передать числа разных типов. Возможен и случай, что одно число будет целым, а второе – вещественным.
    \item Написать программу для определения суммы значений в массиве (целочисленных/вещественных) с использованием шаблона функции.
    \item Реализовать шаблонную функцию, которая считает процент от числа и возвращает значение в программу. И число, и процент передаются как параметры.
    \item С использованием шаблона функции написать программу сортировки методом "пузырька" для массивов величин с  целочисленным, вещественным, символьным и строковыми типами данных. Добавить шаблон функции для вывода массивов на экран.
    \item Написать программу, которая проверяла бы условие равенства/неравенства двух введенных чисел и выводила бы соответствующее сообщение на экран. В программе необходимо объявить и определить шаблон функции \monobf{neq()}, которая будет проверять на неравенство значения различных типов, включая комплексные числа.
\end{enumerate}

\section{Лабораторная работа 5. Векторы}
\begin{enumerate}[leftmargin=*]
    \item Задан массив из  целых чисел. Переместить все минимальные элементы в начало массива, не меняя порядок других. 
    \item Задан целочисленный массив. Определить процентное содержание элементов, превышающих среднеарифметическое всех элементов массива.
    \item Выполнить сортировку массива по возрастанию (убыванию).
    \item Дан массив из 10 элементов. Первые 4 упорядочить по возрастанию, последние 4 по убыванию.
    \item Дан массив 20 целых чисел на отрезке $[-2; 5]$. Упорядочить массив, удалив нули со сдвигом влево, ненулевыми элементами.
    \item Дан массив 15 целых чисел на отрезке $[-5; 5]$. Упорядочить массив, удалив повторяющиеся элементы.
    \item Ввести два массива действительных чисел. Определить максимальные элементы в каждом массиве и поменять их местами. 
    \item Дана квадратная матрица. Вывести на экран элементы, стоящие на диагонали.
    \item Дана матрица. Вывести на экран все нечетные столбцы, у которых первый элемент больше последнего.
    \item Дана матрица $NxM$ случайных чисел. Отсортировать элементы главной диагонали матрицы по убыванию.
    \item Дана матрица $NxM$ случайных чисел. Упорядочить первый столбец матрицы по возрастанию, а последний столбец – по убыванию.
\end{enumerate}

\section{Лабораторная работа 6. Алгоритмы STL}
\begin{enumerate}[leftmargin=*]
    \item Написать программу, которая с помощью алгоритмов STL \monobf{[sort()]} выполняет сортировку массива строк.
    \item Отсортируйте и выведите на экран массив вещественных чисел по возрастанию (убыванию). Для решения этой задачи воспользуйтесь заголовочным файлом \monobf{[\#include <functional>]} и объектами-функциями \monobf{[greater/less]}, напр., \monobf{[sort(array, array+array\_size,greater<double>())]}.
    \item Написать программу, которая с помощью алгоритмов STL \monobf{[next\_permutation()]} выводит на экран все перестановки строки“\textbf{abcd}”.
    \item Написать программу поиска подстроки в строке, основываясь на функциях стандартных библиотек \monobf{[find()]}.
    \item В заданной строке (\monobf{StrText}) заменить все вхождения заданной подстроки (\monobf{StrFind}) на заданную строку (\monobf{StrReplace}). Например, \texttt{StrText=}\monobf{"Informational\\ Technology"}, \texttt{StrFind=} \monobf{"Technology"}, \texttt{StrReplace=}"\monobf{"System"}.
    \item Основываясь на функциях стандартных библиотек написать программу подсчета количества каждого слова, встречающегося в тексте.
    \item Написать шаблонную функцию \monobf{input}, вводящую вектор с клавиатуры.
    \item Написать шаблонную функцию \monobf{part}, принимающую вектор и возвращающую а) первую половину, б) среднюю треть его элементов (тоже как вектор).
    Написать шаблонную функцию \monobf{concat}, принимающую два вектора и возвращающую их сцепление (сначала идут элементы первого, а затем второго).
    \item Написать шаблонную функцию \monobf{repeat}, принимающую вектор \monobf{v} и неотрицательное целое число \monobf{n}, и возвращающую новый вектор, полученный повторением вектора \monobf{v n} раз.
    \item Написать шаблонную функцию \monobf{subseq}, принимающую два вектора и проверяющую, что второй из них является подпоследовательностью первого (например, для векторов (1, 2, 3, 4, 5, 6, 7, 8, 9) и (1, 5, 7, 8) ответ будет \monobf{true}, а для (1, 2, 3, 4, 5, 6, 7, 8, 9) и (2, 1, 5, 7, 8) ответ будет \monobf{false}).
    \item Написать шаблонную функцию \monobf{enlarge}, принимающую вектор по ссылке и вставляющую между каждыми соседними элементами их полусумму.
    \item Написать функцию \monobf{shorten}, принимающую вектор из целых чисел по ссылке и удаляющую элементы, стоящие между соседними (т. е. такими, между которыми нет единиц и двоек) 1 и 2 (именно в таком порядке).
    \item Написать шаблонную функцию \monobf{co}, принимающую два вектора и возвращающую число вхождений второго вектора, как подпоследовательность из элементов, идущих подряд (а не как в задаче 3), в первый (с использованием алгоритма \monobf{search}).
    \item Написать набор шаблонных функций для реализации понятия множества, которое хранится как упорядоченный вектор. Должны быть реализованы следующие операции: ввод, вывод, добавить элемент, удалить элемент, проверить принадлежность элемента множеству, объединение, пересечение и разность.
    \item Написать набор шаблонных функций для реализации понятия многочлена одной переменной, который хранится как вектор коэффициентов. Должны быть реализованы следующие операции: ввод, вывод, задать постоянный многочлен, задать многочлен x, вычислить значение многочлена, сумму, произведение, неполное частное и остаток от деления многочленов (последние две функции предполагают, что коэффициенты принадлежат полю, т. е. можно делить на любой ненулевой коэффициент, и притом точно).
\end{enumerate}
